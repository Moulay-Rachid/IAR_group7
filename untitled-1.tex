\documentclass[12pt]{article}

\begin{document}
\title{Hw1}
\maketitle
\section{l'importance des maison intelegent}
c’est économiser sur le court et le long terme. Car, à partir de votre téléphone, ordinateur, vous pouvez gérer des tâches à distance. En premier lieu, lorsque votre maison intelligente est dotée d’une fonction de détection de présence, elle réduit la consommation d’une ou de plusieurs ampoules. Quand votre maison est vide, les lampes peuvent s’éteindre et se rallumer au moment où quelqu’un entre. En deuxième lieu, une maison connectée vous donne la possibilité de contrôler et de limiter la consommation énergétique de différents appareils.\\
\section{Objectives}
La domotique a commencé avec la possibilité de programmer et de contrôler certains appareils. Au tout début, les connexions entre les différents éléments étaient filaires. Puis il y a eu des avancées majeures. Des éclairages, des appareils électroménagers, des stores ont commencé à être connectés. Maintenant, avec du Wifi ou du Bluetooth par exemple, beaucoup de choses sont possibles. Le plus courant est de connecter son téléphone avec bon nombre d’autres appareils. Vous pouvez bien sûr – en faisant appel à un professionnel – relier tous vos systèmes entre eux : éclairage, chauffage, volets.\\
\end{document}